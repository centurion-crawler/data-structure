\documentclass[UTF8,a4paper]{article}
\usepackage{fancyhdr}
\usepackage{ctex}
\usepackage{amsmath}
\usepackage{listings}
\usepackage{color}
\lstset{ %
	extendedchars=false,            % Shutdown no-ASCII compatible
	language=Matlab,                % choose the language of the code
	basicstyle=\small\sf,    % the size of the fonts that are used for the code
	tabsize=3,                            % sets default tabsize to 3 spaces
	numbers=left,                   % where to put the line-numbers
	numberstyle=\tiny,              % the size of the fonts that are used for the line-numbers
	stepnumber=1,                   % the step between two line-numbers. If it's 1 each line
	% will be numbered
	numbersep=5pt,                  % how far the line-numbers are from the code   %
	keywordstyle=\color[RGB]{33,33,234},               % keywords
	commentstyle=\color[RGB]{0,0,0},    % comments
	stringstyle=\color[rgb]{0.170,0.187,0.102},      % strings
	backgroundcolor=\color{white}, % choose the background color. You must add \usepackage{color}
	showspaces=false,               % show spaces adding particular underscores
	showstringspaces=false,         % underline spaces within strings
	showtabs=false,                 % show tabs within strings adding particular underscores                frame = single,         % adds a frame around the code
	captionpos=b,                   % sets the caption-position to bottom
	breaklines=true,                % sets automatic line breaking
	breakatwhitespace=false,        % sets if automatic breaks should only happen at whitespace
	title=\lstname,                 % show the filename of files included with \lstinputlisting;
	% also try caption instead of title
	mathescape=true,escapechar=?    % escape to latex with ?..?
	escapeinside={\%*}{*)},         % if you want to add a comment within your code
}
\pagestyle{fancy}
\lhead{数据结构作业}
\chead{}
\rhead{\bfseries 22920182204393庄震丰}
\lfoot{}
\cfoot{\thepage}
\rfoot{}
\renewcommand{\headrulewidth}{0.4pt}
\begin{document}
\begin{center}
    \textbf{\LARGE{数据结构作业 第二章、第三章}}\\[0.5cm]
    \normalsize{庄震丰 22920182204393}\\[0.3cm]
    \large{Sept. $30^{th}$, 2019}
\end{center}
\textbf{2-21}\\
    题目要求:将顺序表的元素倒置。\\
    算法分析:直接将头尾元素交换即可。\\
    2-21.cpp
\begin{lstlisting}
#include<bits/stdc++.h>
#define MAXN 1000
using namespace std;
int n;
int a[MAXN];
int main()
{
        cin>>n;
    for (int i=1;i<=n;i++)
        cin>>a[i];
    for (int i=1;i<=n/2;i++)
        swap(a[i],a[n-i+1]);
    for (int i=1;i<=n;i++)
    cout<<a[i]<<" ";
    return 0;
}
\end{lstlisting}
\textbf{2-29}\\
    题目要求:将顺序表A中元素在B和C中的元素删除。\\
    算法分析:对A和B和C顺序表进行遍历,每次指向比A中元素小或者相等的元素,符合判断要求的则标记当前元素,最后再删除。\\
    时间复杂度\textit{O(n)},空间复杂度\textit{O(n)}\\
    2-29.cpp
\begin{lstlisting}
#include<bits/stdc++.h>
#define MAXN 1000
using namespace std;
int a[MAXN]={0},b[MAXN]={0},c[MAXN]={0},a_new[MAXN]={0};
bool dela[MAXN];
void init(int Na,int Nb,int Nc)
{
    for (int i=1;i<=Na;i++)
        cin>>a[i];
    for (int i=1;i<=Nb;i++)
        cin>>b[i];
    for (int i=1;i<=Nc;i++)
        cin>>c[i];
    memset(dela,sizeof(dela),false);
}
int main()
{
    int na,nb,nc;
    cin>>na>>nb>>nc;
    init(na,nb,nc);
    int pb=1,pc=1;
    int cnt=0;
    cout<<endl;
    for (int i=1;i<=na;i++)
    {
        while(b[pb]<a[i]) ++pb;
        while(c[pc]<a[i]) ++pc;
        cout<<i<<" "<<pb<<" "<<pc<<endl;
        if (a[i]==b[pb]&&a[i]==c[pc]) 
            {
                dela[i]=true; //标记删除的点
                cnt++;
            }
    }
    int *p=a+1;
    for (int i=1;i<=na-cnt;i++)
    {
        while (dela[p-a]) p++; 
        a_new[i]=*p;p++;
    }
    for (int i=1;i<=na-cnt;i++)
        cout<<a_new[i]<<" ";
    return 0;
}
\end{lstlisting}
\textbf{2-30}\\
题目要求:在链表数据结构中解决问题。\\
题目分析:建立双向链表,当前元素符合要求时利用pre指针和next指针进行删除当前元素,不用再进行链表头尾特判。\\
时间复杂度\textit{O(n)},空间复杂度\textit{O(n)}\\
2-30.cpp
\begin{lstlisting}
#include<bits/stdc++.h>
#define MAXN 1000
using namespace std;
struct node
{
    int data;
    node* next;
    node* pre;
};
node *heada,*headb,*headc;
node* build(int N)
{
    int tmp;
    node* head;
    node* p1;
    node* p2;
    head=(node *)malloc(sizeof(node));
    cin>>tmp;
    head->data=tmp;
    head->pre=NULL;
    p1=head;
    for (int i=1;i<=N-1;i++)
    {
        p2=(node *)malloc(sizeof(node));
        cin>>tmp;
        p2->data=tmp;
        p1->next=p2;
        p2->pre=p1;
        p1=p2;
    }
    p1->next=NULL;
    return head;
    }
void lstdel()
{
    node *pa=heada,*pb=headb,*pc=headc;
    while (pa!=NULL)
    {
            while(pb->data<pa->data&&pb->next!=NULL) pb=pb->next;
            while(pc->data<pa->data&&pc->next!=NULL) pc=pc->next;
            if (pa->data==pb->data&&pa->data==pc->data)
            {
                if (pa==heada) 
                {
                    heada->next->pre=NULL;
                    node *ha=heada;
                    heada=pa->next;
                    pa=pa->next;
                    free(ha);
                }
                else if (pa->next==NULL)
                {
                    pa->pre->next=NULL;
                    free(pa);
                  break;
             }
                else
            {
                node *p=pa;
                pa->pre->next=pa->next;
                pa->next->pre=pa->pre;
                pa=pa->next;
                free(p);
            }            
        }
        else
        {
           pa=pa->next;
        }        
    }
}
void getlst(node * Ha)
{
    while(Ha!=NULL)
    {
        printf("%d ",Ha->data);
        Ha=Ha->next;
    }
}
int main()
{
    int na,nb,nc;
    cin>>na>>nb>>nc;
    heada=build(na);
    headb=build(nb);
    headc=build(nc);
    lstdel();
    getlst(heada);
    return 0;
}
\end{lstlisting}
\textbf{2-39}\\
题目描述:计算稀疏多项式。\\
算法分析;先计算好$x..x^{e(m)}$,再针对每项进行相加操作。\\
时间复杂度\textit{O(n)},空间复杂度\textit{O(n)}\\
2-39.cpp
\begin{lstlisting}
#include<bits/stdc++.h>
#define MAXN 100
using namespace std;
long long x[100]={1};
long long c[100]={0};
long long e[100]={0};
int main()
{
    long long p=0,x0,m;
    cin>>x0>>m;
    for (int i=1;i<=m;i++)
        cin>>c[i];
    for (int i=1;i<=m;i++)
        cin>>e[i];
    for (int i=1;i<=e[m];i++)
        x[i]=x[i-1]*x0;
    for (int i=1;i<=m;i++)
        p+=c[i]*x[e[i]];
    cout<<p;
    return 0;
}
\end{lstlisting}
\textbf{3-15}\\
题目要求:双向栈,实现初始化,push和pop操作。\\
算法分析:模拟单向栈,当push是0栈和1栈重合,或者栈为空时,返回ERROR。\\
时间复杂度\textit{O(n)},空间复杂度\textit{O{n}}.
3-15.cpp
\begin{lstlisting}
    #include<bits/stdc++.h>
#define MAXN 100
using namespace std;
int stacklength;
int a[MAXN]={0};//栈数组
int *p0,*p1;//栈顶指针
void inistack()//初始化栈
{
    memset(a,sizeof(a),0);
    p0=a,p1=a+stacklength;
}
int pushtwi(int i,int x)//在栈顶插入元素
{
    if (i)
    {
        if (p1-p0>1)
            {
                p1--;
                *p1=x;
                return 1;
            }
        else return -1; 
    }
    else
    {
        if (p1-p0>1)
        {
            p0++;
            *p0=x;
            return 1;
        }
        else return -1;
    }
    
}
int poptwi(int i)//弹出栈顶元素
{
    if (i)
    {
        if (a+stacklength>p1)
        {
            int num=*p1;
            *p1=0;
            p1++;
            return num;
        }
        else return -1;
    }
    else 
    {
        if (p0>a)
        {
            int num=*p0;
            *p0=0;
            p0--;
            return num;
        }
        else return -1;
    }
}
void opstack()//pop或者push操作
{
    int opnum;
    int statupush,statupop,I,X,op;
    cout<<"input the operator times:";
    cin>>opnum;
    for (int i=1;i<=4;++i)
    {
        cin>>I>>X>>op;
        if (op) 
            {
                statupush=pushtwi(I,X);cout<<statupush<<" ";
            }
        else 
            {
                statupop=poptwi(I);
                cout<<statupop<<" ";
            }
    }
}
int main()
{
    cin>>stacklength;
    inistack();
    return 0;
}
\end{lstlisting}
\textbf{3-19}\\
题目要求:括号序列匹配\\
算法分析:建立栈,当栈为空,或者当前元素与栈顶元素不匹配就push,否则元素不插入,并且pop栈顶元素\\
时间复杂度\textit{O(n)},空间复杂度\textit{O(n)}.\\
3-19.cpp
\begin{lstlisting}
    #include<bits/stdc++.h>
using namespace std;
stack <int> a;//定义栈
bool match_braket(char s1,char s2)//判断栈顶元素与当前元素是否匹配
{
    if (s1=='{'&& s2=='}') return true;
    if (s1=='['&& s2==']') return true;
    if (s1=='('&& s2==')') return true;
    return false;
}
int main()
{
    string c;
    cin>>c;
    while(!a.empty())
        a.pop();
    for (int i=0;i<c.length();i++)
    {
        if (a.size()==0) 
            {
                a.push(c[i]);//若栈为空则直接push
                continue;
            }
        if (match_braket(a.top(),c[i]))//如果当前元素栈顶元素匹配则弹出
            a.pop();
            else a.push(c[i]);
    }
    if (!a.empty())
        cout<<"The brakets don't match";
        else cout<<"The brakets match";
return 0;
}
\end{lstlisting}
\textbf{3-28}\\
题目要求:实现循环链表队列的init,push,pop操作。\\
算法分析:首先建立循环链表,给定head和rear指针,当插入元素数目超过给定范围时,从rear指针下删除队尾元素。\\
时间复杂度\textit{O(n)},空间复杂度\textit{O(n)}.\\
3-28.cpp
\begin{lstlisting}
    #include <stdio.h>
    #include <stdlib.h>
    #define ERROR 0
    #define OK 1
    #define OVERFLOW 0
    typedef int qelemType;
    typedef struct queue
    {
        qelemType data;
        struct queue *next;
    }queue,*linkqueue;
    typedef struct
    {
        linkqueue rear;
        int length;
    }sqqueue;
    void initQueue(sqqueue &queue)//置空队列
    {
        queue.rear=(linkqueue)malloc(sizeof(queue));
        queue.rear->next=queue.rear;
    }
    
    int emptyQueue(sqqueue &queue)//判队列是否为空
    {
        if(queue.rear->next==queue.rear)
            return OK;
        else
            return 0;
    }
    int enqueue(sqqueue &queue,qelemType e)
    {
        linkqueue p;
        p=(linkqueue)malloc(sizeof(queue));
        if(!p)
            return OVERFLOW;
        p->data=e;
        p->next=queue.rear->next;
        queue.rear->next=p;
        queue.rear=p;
        return OK;
    }
    int delqueue(sqqueue &queue,qelemType &e)
    {
        linkqueue p;
        if(queue.rear->next==queue.rear)
            return ERROR;//若队列为空返回0
        p=queue.rear->next->next;//循环链表队列队尾指针下一结点(也即头结点)的下一结点(即队头指针)
        e=p->data;
        queue.rear->next->next=p->next;
        free(p);
        return OK;
    }
    int main()
    {
        sqqueue queue2;
        qelemType num;
        initQueue(queue2);
        if(emptyQueue(queue2))
            printf("该队列目前为空!\n");
        else
            printf("该队列不为空!\n");
        for(int i=1;i<=10;i++)
            if(enqueue(queue2,i))
                printf("元素%d成功入列!\n",i);
        printf("\n\n");
        for(int j=1;j<=9;j++)
            if(delqueue(queue2,num))
                printf("元素%d成功出列!\n",num);
        if(emptyQueue(queue2))
            printf("该队列目前为空!\n");
        else
            printf("该队列不为空!\n");
        return 0;
    }
\end{lstlisting}
\textbf{3-32}\\
题目要求:k阶斐波那契数列。\\
算法分析:首先明确k阶斐波那契数列的定义。
\begin{equation}
    \begin{aligned}
    f_0=0,f_1=0,f_2=0,&\cdots,f_{k-2}=0,f_{k-1}=0,f_k=1.\\
    f_i&=\sum_{j=i-k-1}^{i-1} f_j\\
    f_{i-1}&=\sum_{j=i-k-2}^{i-2} f_j\\
    \end{aligned}
\end{equation}
所以,可以递推得到,
\begin{equation}
    \begin{aligned}
    f_i=2\cdot f_{i-1}-f_{i-1-k}
    \end{aligned}
\end{equation}
而$f_{i-1}$与$f_{i-1-k}$分别是队头和队尾元素。用封装好的队列push和pop函数即可。\\
时间复杂度\textit{O(n)},空间复杂度即队列复杂度\textit{O(n)}.\\
3-32.cpp
\begin{lstlisting}
#include <bits/stdc++.h>
using namespace std;
queue <int> a;
int main()
{
    int n,k;
    \\k阶斐波那契f0=0,f1=0,...fk-2=0,fk-1=0;
    for (int i=0;i<k-1;i++)
        a.push(0);
        a.push(1);
    for (int i=k;i<=n;i++)
        {
            a.push(2*a.front()-a.back());//fi=2*f(i-1)-f(i-k-1) 分别为队头和队尾
            if (a.size()>k) a.pop();
        }
    cout<<a.front();
}
\end{lstlisting}
\newpage
\begin{center}
    \Large{马踏棋盘解题报告}\\
    \large{2019.09.30}
\end{center}
\section{题目描述}
将马随机放置再国际象棋8$times$8棋盘Board[8][8]的某个方格中,马按走棋的规则进行移动。
要求每个方格只进入一次,走遍棋盘上全部64个方格,求出马的行走路线,并按要求输出行走路线,
将1,2,$\cdots$,64依次填入一个8$times$8的方阵,输出之。
\subsection{测试数据}
自行指定一个马的初始位置(i,j),得到路径结果。
\subsection{实现提示}
一般来说,马位于(i,j)时,可以走到以下八个位置之一。
$$
    (i-2,j-1),(i-2,j+1),(i+2,j-1),(i+2,j+1),(i-1,j+2),(i-1,j-2),(i+1,j+2),(i+1,j-2)
$$
但是,当以上位置超出边界或者已经走过,就只能选择回溯。
\section{算法分析}
    栈操作(深搜)+贪心算法。\\

    每到达一个点试探八个方向是否能走,对于能进行的方向判断该方向能够继续走的方向,
    取其中子方案数最小的方向进行,这样最容易到达搜索树的根节点。\\
    直到能够走完棋盘为止。
\section{程序及运行结果}
\subsection{程序代码}
horse.cpp
\begin{lstlisting}
#include <bits/stdc++.h>
#define N 8
#define INF 999999999
using namespace std;

int dx[9]={0,2,2,-2,-2,1,1,-1,-1};
int dy[9]={0,-1,1,1,-1,-2,2,-2,2};
bool MAP[N+1][N+1];
int sign[N+1][N+1]={0};
void print()
{
    for (int i=1;i<=N;i++)
        {
            for (int j=1;j<=N;j++)
                cout<<sign[i][j]<<" ";
        cout<<endl;
        }
}
bool judge(int i,int j)
{
    if (!MAP[i][j]&&i>0&&i<N+1&&j>0&&j<N+1)
    return true;
    else return false;
}
int judgepath(int x_,int y_)
{
	int c;
	int tag=0;
	for (c = 0; c < 8; c++)
	{
		if (judge(x_ + dx[c], y_ + dy[c]))
		{
			tag++;
		}
	}
	return tag;
}
void walk(int x,int y,int cnt)
{
    cout<<x<<" "<<y<<endl;
    if (cnt==N*N)
        {
            print();
            exit(0);
        }
    int MIN=INF,rec=-1,rec2=0;
    for (int i=1;i<=8;i++)
    {
        if (!judge(x+dx[i],y+dy[i]))
              continue;
        int path=judgepath(x+dx[i],y+dy[i]);  
            if (MIN>path)
            {
                MIN=path;
                if (rec) rec2=rec;
                rec=i;
            }
    }
    if (MIN!=INF)
    {
                sign[x+dx[rec]][y+dy[rec]]=cnt+1;
                MAP[x+dx[rec]][y+dy[rec]]=true;
                walk(x+dx[rec],y+dy[rec],cnt+1);
                MAP[x+dx[rec]][y+dy[rec]]=false;
                sign[x+dx[rec]][y+dy[rec]]=0;
    } 
        return;
}
void init()
{
    int x0,y0;
    memset(MAP,sizeof(MAP),false);
    cin>>x0>>y0;
    sign[x0][y0]=1;
    MAP[x0][y0]=true;
    walk(x0,y0,1);
}
int main()
{
    init();
    return 0;
}
\end{lstlisting}
\subsection{样例输入输出}
\noindent\textbf{样例输入1}:\\
2 3\\
\noindent\textbf{样例输出1}:\\
2~ 57 30 37 16 47 14 39\\
29 36 1~ 56 61 38 17 46\\
58 3~ 60 31 48 15 40 13\\
35 28 53 62 55 64 45 18\\
4~ 59 34 49 32 19 12 41\\
27 52 25 54 63 44 9~ 20\\
24 5~  50 33 22 7~ 42 11\\
51 26 23 6~ 43 10 21 8\\
\noindent\textbf{样例输入2}:\\
3 4\\
\noindent\textbf{样例输出2}:\\
36 45 16 13 22 59 18 11\\
15 2~ 37 44 17 12 21 60\\
46 35 14 1~ 58 23 10 19\\
3~ 38 51 54 43 20 61 24\\
34 47 42 57 52 55 28 9~\\
39 4~ 53 50 31 62 25 64\\
48 33 6~ 41 56 27 8~ 29\\
5~ 40 49 32 7~ 30 63 26\\
\section{总结与思考}
对于没有优化的搜索,不难求得时间复杂度达到$8^{64}$,显然难以在短时间得到结果。
因此贪心优化利用后继数最少压缩搜索量,较快得到一组可行解,但是得到的解数目有限,由于没有完全搜索,
这是本算法的弊端所在。
\end{document}
