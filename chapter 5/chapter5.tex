\documentclass[UTF8,a4paper]{article}
\usepackage{fancyhdr}
\usepackage{ctex}
\usepackage{CJK}
\usepackage{amsmath}
\usepackage{listings}
\usepackage{color}
\usepackage{xcolor}
\usepackage{geometry}
\usepackage{indentfirst}
\setlength{\parindent}{2em}
\geometry{left=1.5cm,right=1.5cm,top=1.5cm,bottom=1.5cm}
\lstset{breaklines}%这条命令可以让LaTeX自动将长的代码行换行排版
\lstset{extendedchars=false}%这一条命令可以解决代码跨页时,章节标题,页眉等汉字不显示的问题
\lstset{ 
	language=C++,                % choose the language of the code
	basicstyle=\small\sf,    % the size of the fonts that are used for the code
	tabsize=3,                            % sets default tabsize to 3 spaces
	numbers=left,                   % where to put the line-numbers
	numberstyle=\tiny,              % the size of the fonts that are used for the line-numbers
	stepnumber=1,                   % the step between two line-numbers. If it's 1 each line
	% will be numbered
	numbersep=5pt,                  % how far the line-numbers are from the code   %
	keywordstyle=\color[RGB]{33,33,234},               % keywords
	commentstyle=\color[RGB]{0,0,0},    % comments
	stringstyle=\color[rgb]{0.170,0.187,0.102},      % strings
    backgroundcolor=\color{white},
    rulesepcolor=\color[RGB]{20,20,20},  % choose the background color. You must add \usepackage{color}
	showspaces=false,               % show spaces adding particular underscores
	showstringspaces=false,         % underline spaces within strings
	showtabs=false,                 % show tabs within strings adding particular underscores                frame = single,         % adds a frame around the code
	captionpos=b,                   % sets the caption-position to bottom
	breaklines=true,                % sets automatic line breaking
	breakatwhitespace=false,        % sets if automatic breaks should only happen at whitespace
	title=\lstname,                 % show the filename of files included with \lstinputlisting;
	% also try caption instead of title
	mathescape=true,escapechar=?    % escape to latex with ?..?
	escapeinside={\%*}{*)},         % if you want to add a comment within your code
	%columns=fixed,                  % nice spacing
	%morestring=[m]',                % strings
	%morekeywords={%,...},%          % if you want to add more keywords to the set
	%    break,case,catch,continue,elseif,else,end,for,function,global,%
	%    if,otherwise,persistent,return,switch,try,while,...},%
}
\pagestyle{fancy}
\lhead{数据结构作业}
\chead{}
\rhead{\bfseries 22920182204393庄震丰}
\lfoot{}
\cfoot{\thepage}
\rfoot{}
\renewcommand{\headrulewidth}{0.4pt}
\begin{document}
\begin{center}
    \textbf{\LARGE{数据结构作业 第二章、第三章}}\\[0.5cm]
    \normalsize{庄震丰 22920182204393}\\[0.3cm]
    \large{Oct. $19^{th}$, 2019}
\end{center}
\textbf{5-18}\\
    题目要求:将顺序表的元素右移k位,只使用一个附加存储,交换\\
    算法分析:考虑一个顺序表,每个元素和前面一个元素交换直到再次访问到开始节点,这样的效果和每个元素右移一位一样。\\
    因此我们考虑每次元素和k位之前的元素交换,这里只需一个临时储存变量。而对于一个n长度,k移动要求的顺序表,将会形成gcd(n,k)个移动闭环,每个环长$\frac{n}{k}$。因此每个元素只交换了一次。
    时间复杂度\textit{O}(n),附加空间\textit{O}(1)。
    5-18.cpp
\begin{lstlisting}
#include<bits/stdc++.h>
using namespace std;
int n,k;
int A[10000];
int tmp;
void change(int x,int y)
{
    int l=min(y-x,n-y);
    for (int i=0;i<l;i++)
        {   
            tmp=A[x+i];
            A[x+i]=A[y+i];
            A[y+i]=tmp;
        }
}
int gcd (int x,int y)
{
    int z=x%y;
    while(z!=0)
    {
        x=y;
        y=z;
        z=x%y;
    }
    return y;
}
int main()
{
    cin>>n>>k;
    for (int i=0;i<n;i++)
        cin>>A[i];
    for (int i=0;i<gcd(n,k);i++)
        {
            for (int j=i;j!=(i+k)%n;j=(j-k+n)%n)
            {
                tmp=A[j];
                A[j]=A[(j-k+n)%n];
                A[(j-k+n)%n]=tmp;
            }
        }
    for (int i=0;i<n;i++)
        cout<<A[i]<<" ";
    return 0;
}
\end{lstlisting}
\textbf{5-19}\\
    题目要求:求矩阵的马鞍点\\
     算法分析:根据马鞍点的定义, 马鞍点一定等于其所在行的最小值,为其所在行的最大值点,因此用两个辅助数组max,min记录当前行的最小值位置和列的最大值位置。\\
    首先建立矩阵,得到每行最小值点位置,存入$max[i][cnt]$中。\\
    第二次扫描矩阵,得到每列最大值点位置,存入$max[j][cnt]$中。\\
    第三次扫描$min[i][cnti]$,判断每行$max[min[i][cnti]][cntj]$是否有等于i的点,有就输出没有就i++\\
    若每行每列最多一个马鞍点,时间复杂度复杂度\textit{O($n^2$)}.\\
    每行每列多个马鞍点,时间复杂度\textit{O($n^3$)}\\
    空间复杂度\textit{O($n^2$)}
    5-19.cpp
\begin{lstlisting}
#include <bits/stdc++.h>
#define MAX 256 
typedef struct
{
	int x;
	int y;
	int value;
}Node;
typedef Node SNode;
bool Algo_5_19(int a[MAX][MAX], int row, int col, SNode p[MAX]);
void Min_row(int a[MAX][MAX], int col, int i, Node min[MAX]);
bool IsMax_col(int a[MAX][MAX], int row, Node v);
int main(int argc, char *argv[])
{
	SNode p[MAX];	
	int row,col
	int a[MAX][MAX] ;
	int i, j, k;
	cin>>row,col;
	for(i=0; i<row; i++)
	{
		for(j=0; j<col; j++)
			scanf("%3d ",&a[i][j]);
	}
	
	
	if(Algo_5_19(a, row, col, p))
	{
		printf("此矩阵中存在马鞍点...\n");
		for(k=1; k<=p[0].value; k++)
			printf("第 %d 行第 %d 列的马鞍点 %d\n", p[k].x, p[k].y, p[k].value);
	}
	else
		printf("此矩阵中不存在马鞍点...\n");
	
	printf("\n");
	
	return 0;
}
bool Algo_5_19(int a[MAX][MAX], int row, int col, SNode p[MAX])
{
	int i, k;
	Node min[MAX];
	
	p[0].value = 0;
	
	for(i=0; i<row; i++)
	{
		Min_row(a, col, i, min);
		for(k=1; k<=min[0].value; k++)
		{
			if(IsMax_col(a, row, min[k]))
			{
				p[0].value++;
				p[p[0].value] = min[k];
			}
		}
	}
	
	if(p[0].value)
		return true;
	else
		return false;	
}
void Min_row(int a[MAX][MAX], int col, int i, Node min[MAX])
{
	int j;
	
	min[0].value = 0;
	
	for(j=0; j<col; j++)
	{
		if(!j || a[i][j]==min[j-1].value)
			min[0].value++;
		else
		{
			if(a[i][j]<min[j-1].value)
				min[0].value = 1;
			else
				continue;
		}
			
		min[min[0].value].x = i;
		min[min[0].value].y = j;
		min[min[0].value].value = a[i][j];			
	}
}

bool IsMax_col(int a[MAX][MAX], int row, Node v)
{
	int i;
	
	for(i=0; i<row; i++)
	{
		if(a[i][v.y]>v.value)
			return false;
	}
	
	return true;
}

\end{lstlisting}
\end{document}
