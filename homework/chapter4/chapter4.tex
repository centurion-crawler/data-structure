\documentclass[UTF8,a4paper]{article}
\usepackage{fancyhdr}
\usepackage{ctex}
\usepackage{CJK}
\usepackage{amsmath}
\usepackage{listings}
\usepackage{graphics}
\usepackage{graphicx}
\usepackage{color}
\usepackage{xcolor}
\usepackage{geometry}
\usepackage{indentfirst}
\setlength{\parindent}{2em}
\geometry{left=1.5cm,right=1.5cm,top=2cm,bottom=1.5cm}
\lstset{breaklines}%这条命令可以让LaTeX自动将长的代码行换行排版
\lstset{extendedchars=false}%这一条命令可以解决代码跨页时,章节标题,页眉等汉字不显示的问题
\lstset{ 
	language=C++,                % choose the language of the code
	basicstyle=\small\sf,    % the size of the fonts that are used for the code
	tabsize=3,                            % sets default tabsize to 3 spaces
	numbers=left,                   % where to put the line-numbers
	numberstyle=\tiny,              % the size of the fonts that are used for the line-numbers
	stepnumber=1,                   % the step between two line-numbers. If it's 1 each line
	% will be numbered
	numbersep=5pt,                  % how far the line-numbers are from the code   %
	keywordstyle=\color[RGB]{33,33,234},               % keywords
	commentstyle=\color[RGB]{0,0,0},    % comments
	stringstyle=\color[rgb]{0.170,0.187,0.102},      % strings
    backgroundcolor=\color{white},
    rulesepcolor=\color[RGB]{20,20,20},  % choose the background color. You must add \usepackage{color}
	showspaces=false,               % show spaces adding particular underscores
	showstringspaces=false,         % underline spaces within strings
	showtabs=false,                 % show tabs within strings adding particular underscores                frame = single,         % adds a frame around the code
	captionpos=b,                   % sets the caption-position to bottom
	breaklines=true,                % sets automatic line breaking
	breakatwhitespace=false,        % sets if automatic breaks should only happen at whitespace
	title=\lstname,                 % show the filename of files included with \lstinputlisting;
	% also try caption instead of title
	mathescape=true,escapechar=?    % escape to latex with ?..?
	escapeinside={\%*}{*)},         % if you want to add a comment within your code
	%columns=fixed,                  % nice spacing
	%morestring=[m]',                % strings
	%morekeywords={%,...},%          % if you want to add more keywords to the set
	%    break,case,catch,continue,elseif,else,end,for,function,global,%
	%    if,otherwise,persistent,return,switch,try,while,...},%
}
\pagestyle{fancy}
\lhead{数据结构作业}
\chead{}
\rhead{\bfseries 22920182204393庄震丰}
\lfoot{}
\cfoot{\thepage}
\rfoot{}
\renewcommand{\headrulewidth}{0.4pt}
\begin{document}
\begin{center}
    \textbf{\LARGE{数据结构作业 第四章——串}}\\[0.5cm]
    \normalsize{庄震丰 22920182204393}\\[0.3cm]
    \large{Nov. $19^{th}$, 2019}
\end{center}
\textbf{4-12}\\
题目要求:编写一个实现串的置换操作Replace(\&S,T,V)的算法。\\
算法分析:由于与4-17一样,这里采用普通算法求解子串位置,即一个个位置相比。\\
空间复杂度O(N+M)=O(N),时间复杂度O(NM)。\\
\textbf{4-12}\\
\begin{lstlisting}
    //常规方法替换子串
    #include<bits/stdc++.h>
    using namespace std;
    #define maxn 1000
    string S;
    void Replace(string &S,string T, string V)
    {
        int flag[maxn]={0};
        int p=0,d=V.length()-T.length();
        for (int i=0;i<S.length();i++)
        {
            if (S.substr(i,T.length())==T)
                flag[++p]=i+1+d*(p-1);
        }
        for (int i=1;i<=p;i++)
            S.replace(flag[i]-1,T.length(),V);
    }
    int main()
    {
        string t,v;
        int operatenum;
        cout<<"Please input the operaternumber:"<<endl;
        cin>>operatenum;
        getchar();
        getline(cin,S);
        for (int i=1;i<=operatenum;i++)
        {
            getline(cin,t);
            getline(cin,v);
            Replace(S,t,v);
        }
        cout<<S;
        return 0;
    }
\end{lstlisting}
\textbf{4-17}\\
题目要求:编写算法,实现串的置换操作Replace(\&S,T,V)。\\
算法分析:采用多次匹配的KMP算法,求解所有匹配点,当且仅当当前模式串在源串中没有重叠,若有重叠只考虑第一个。\\
先求nextval数组,求解完后,利用nextval数组进行匹配,匹配失败的时候直接把当前点j跳到模式串中的nextval[j],去匹配,减少无效匹配次数。\\
空间复杂度O(N),匹配时间复杂度O(N+M),初始化为O(M),总体实现是线性时间。\\
4-17.cpp\\
\begin{lstlisting}
    #include<bits/stdc++.h>
    using namespace std;
    #define maxn 1000
    string S;
    int n;
    void getnext(string b, int *nextval)//求next数组
    {
        int len = b.length();
        int j = 0 , k = -1 ;
        nextval[0] = -1 ;
        while(j < len)//查找多次且可重叠时len不能减一,因为该单词的末尾加一的next也需要被下一次查询用到。
        {
            if(k == -1 || b[k] == b[j])
            {
                k++;
                j++;
                // 下面nest数组的优化
                if(b[k] != b[j])
                    nextval[j] = k ;
                else
                    nextval[j] = nextval[k];
            }
            else
            {
                k = nextval[k];
            }
        }
    }
    void Replace(string &S,string b,string a)
    {//子串替换
        int flag[maxn]={0};
        
        getline(cin,S);
            int nextval[10009];
            getline(cin,b);
            getline(cin,a);
            getnext(b, nextval);
    
            int lena = a.length() , lenb = b.length();
            
            int i = 0 , j = 0;
            int ans = 0 ,d=lena-lenb;
            while(i < lena)
            {
                if(j == -1 || S[i] == b[j])
                {
                    i++ ;
                    j++ ;
                }
                else
                {
                    j = nextval[j];
                }
                if(j == lenb)
                {
                    flag[++ans]=i+1+d*(ans-1);
                    j = nextval[j] ;
                }
            }
            printf("%d\n" , ans);
        for (int i=1;i<=ans;i++)
            S.replace(flag[i]-1,lenb,a);
    }
    int main()
    {
        string A,B;
        cout<<"please input the S string:"<<endl;
        getline(cin,S);//输入源串
        cout<<"please input the operate number:"<<endl;
        cin>>n;//输入操作总数
        getchar();
        for (int i=1;i<=n;i++)
        {
            getline(cin,A);
            getline(cin,B);
            Replace(S,A,B);
        }
        return 0 ;
    }
\end{lstlisting}
\end{document}
