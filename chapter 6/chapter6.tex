\documentclass[UTF8,a4paper]{article}
\usepackage{fancyhdr}
\usepackage{ctex}
\usepackage{CJK}
\usepackage{amsmath}
\usepackage{listings}
\usepackage{graphics}
\usepackage{graphicx}
\usepackage{color}
\usepackage{xcolor}
\usepackage{geometry}
\usepackage{indentfirst}
\setlength{\parindent}{2em}
\geometry{left=1.5cm,right=1.5cm,top=2cm,bottom=1.5cm}
\lstset{breaklines}%这条命令可以让LaTeX自动将长的代码行换行排版
\lstset{extendedchars=false}%这一条命令可以解决代码跨页时,章节标题,页眉等汉字不显示的问题
\lstset{ 
	language=C++,                % choose the language of the code
	basicstyle=\small\sf,    % the size of the fonts that are used for the code
	tabsize=3,                            % sets default tabsize to 3 spaces
	numbers=left,                   % where to put the line-numbers
	numberstyle=\tiny,              % the size of the fonts that are used for the line-numbers
	stepnumber=1,                   % the step between two line-numbers. If it's 1 each line
	% will be numbered
	numbersep=5pt,                  % how far the line-numbers are from the code   %
	keywordstyle=\color[RGB]{33,33,234},               % keywords
	commentstyle=\color[RGB]{0,0,0},    % comments
	stringstyle=\color[rgb]{0.170,0.187,0.102},      % strings
    backgroundcolor=\color{white},
    rulesepcolor=\color[RGB]{20,20,20},  % choose the background color. You must add \usepackage{color}
	showspaces=false,               % show spaces adding particular underscores
	showstringspaces=false,         % underline spaces within strings
	showtabs=false,                 % show tabs within strings adding particular underscores                frame = single,         % adds a frame around the code
	captionpos=b,                   % sets the caption-position to bottom
	breaklines=true,                % sets automatic line breaking
	breakatwhitespace=false,        % sets if automatic breaks should only happen at whitespace
	title=\lstname,                 % show the filename of files included with \lstinputlisting;
	% also try caption instead of title
	mathescape=true,escapechar=?    % escape to latex with ?..?
	escapeinside={\%*}{*)},         % if you want to add a comment within your code
	%columns=fixed,                  % nice spacing
	%morestring=[m]',                % strings
	%morekeywords={%,...},%          % if you want to add more keywords to the set
	%    break,case,catch,continue,elseif,else,end,for,function,global,%
	%    if,otherwise,persistent,return,switch,try,while,...},%
}
\pagestyle{fancy}
\lhead{数据结构作业}
\chead{}
\rhead{\bfseries 22920182204393庄震丰}
\lfoot{}
\cfoot{\thepage}
\rfoot{}
\renewcommand{\headrulewidth}{0.4pt}
\begin{document}
\begin{center}
    \textbf{\LARGE{数据结构作业 第六章}}\\[0.5cm]
    \normalsize{庄震丰 22920182204393}\\[0.3cm]
    \large{Oct. $25^{th}$, 2019}
\end{center}
\textbf{6-33}\\
    题目要求:假定用两个一维数组作为有n个结点的二叉树的存储结构,L[i]和R[i]分别指示结点i(i=1,2,...n)的左孩子和右孩子,0表示空,试写出一个算法判断u是否为v的子孙。\\
	算法分析:对于给定的u和v,从v开始向下递归搜索每个结点看是否为u,并用全局变量进行标记,如果v的子孙没有则会全部遍历,若找到则直接输出判断结果。
	\\
	时间复杂度为O(N),空间复杂度O(N).\\
	6-33.cpp
\begin{lstlisting}
	#include<bits/stdc++.h>
	using namespace std;
	#define maxn 1000
	int L[maxn+1]={0},R[maxn+1]={0};//建立L,R数组储存节点左右儿子
	bool flag=false;
	void getfa(int x,int target)//从v开始往下递归搜素儿子节点
	{
		if (x==target) {flag=true;return;}
			if (L[x]!=0) getfa(L[x],target);
			if (R[x]!=0) getfa(R[x],target);
	}
	int main()
	{
		int n;
		int v,u;
		cin>>n;
		for (int i=1;i<=n;i++)
			cin>>L[i]>>R[i];
		cin>>v>>u;
		getfa(v,u); 
		cout<<flag;
		return 0;
	}
\end{lstlisting}
\textbf{样例输入1}\\
12\\
2 3\\
6 7\\
8 9\\
10 11\\
12 0\\
0 0\\
0 0\\
0 0\\
0 0\\
0 0\\
0 0\\
2 10\\
\textbf{样例输出1}\\
1\\
\textbf{说明}\\
\includegraphics[width=0.7\textwidth]{6-33.png}\\
输入判断2是否为10的祖先,由图可知为真,输出1。\\\\
\textbf{6-62}\\
    题目要求:试编写算法,求一棵以孩子-兄弟表示的树,编写其深度的算法。\\
    算法分析:输入一棵双亲表示的树,运用机构提结点将其转换成孩子-兄弟表示法,再将其进行深搜,每从左结点递归则层数加一,最后统计最大值即可。\\
	时间复杂度O(n),空间复杂度O(n)。\\
	6-62.cpp
\begin{lstlisting}
	#include<bits/stdc++.h>
	using namespace std;
	#define maxn 1000
	#define inf 0x3f3f3f3f
	struct node
	{
		char name;
		node * lson;
		node * bro;
	};//孩子兄弟结构结点定义
	struct parentnode
	{
		int fa;
		char c;
		node * point;
	};//双亲数组结构结点定义
	parentnode A[maxn];
	int n;
	int cnt=1;
	node* trans()//将双亲数组结构转变成孩子兄弟树形结构
	{
		node * hp,*p1,*p2;
		hp=(node*)malloc(sizeof(node));
		hp->name=A[1].c;
		hp->bro=NULL;
		A[1].point=hp;//根节点单独判断
		for (int i=2;i<=n;i++)
			{
				if (A[i].fa!=A[i-1].fa) 
				{
					p1=(node*)malloc(sizeof(node));
					p1->bro=NULL;
					p1->lson=NULL;
					p1->name=A[i].c;
					A[i].point=p1;
					A[A[i].fa].point->lson=p1;
					//当前结点是新一层
				}
				else 
				{
					p2=(node *)malloc(sizeof(node));
					p2->bro=NULL;
					p2->lson=NULL;
					p2->name=A[i].c;
					A[i].point=p2;
					p1->bro=p2;
					p1=p2;指针移动到兄弟结点
					//当前结点层数不变
				}
			}
		return hp;
	}
	void finddeep(node * p,int deep)//按照孩子-兄弟结构的指针进行遍历即可
	{
		cout<<p->name<<" ";
		if (p->bro==NULL&&p->lson==NULL) 
			{
				cnt=max(cnt,deep);
				
				return;
			}
		if (p->lson!=NULL) {finddeep(p->lson,deep+1);cout<<p->name<<" ";}
		if (p->bro!=NULL) {finddeep(p->bro,deep);cout<<p->name<<" ";}
		return;
	}
	bool cmp(parentnode a, parentnode b)
	{
		if (a.fa<b.fa) return true;
		else return false;
	}
	int main()
	{
		node * Heap;
		cin>>n;
		A[0].fa=-inf;
		for (int i=1;i<=n;i++)
		{ 
			scanf("%d %c",&A[i].fa,&A[i].c);//输入每个结点的父亲信息和该点的标号
		}
		sort(A+1,A+n+1,cmp);
			Heap=trans();//返回根节点指针
			finddeep(Heap,1);
			cout<<endl<<cnt;
		return 0;
	}
\end{lstlisting}
\textbf{样例输入一}\\
10\\
R -1\\
A 1\\
B 1\\
C 1\\
D 2\\
E 2\\
F 4\\
G 7\\
H 7\\
K 7\\
\textbf{样例输出一}\\
4\\
\textbf{样例说明}\\
\includegraphics[width=0.5\textwidth]{6-62.png}
\includegraphics[width=0.5\textwidth]{6-62-1.png}
样例描述的树和孩子-兄弟表示法如图所示,层数为4。\\
\newpage
\noindent\textbf{6-65}\\
题目描述:已知一棵二叉树的前序和中序序列分别储存在两个一维数组中,试编写算法建立改树的二叉链表。\\
算法分析:先求树的结构,将其储存在一个数组中,在通过这个数组将其转变为链表形式,最后用后序遍历验证输出。\\
时间复杂度O(n),空间复杂度O(n)。\\
6-65.cpp
\begin{lstlisting}
	#include<bits/stdc++.h>
	using namespace std;
	#define maxn 1000
	#define inf 0x3f3f3f3f\\设置无穷大
	struct node
	{
		char c;
		node *lson;
		node *rson;
	};\\二叉链表结构体数组
	int n;
	node * hp;
	char Tree[maxn]={0};
	void work(string ptree,string mtree,int cnt)\\将二叉树两种遍历转变成数组存储
	{
		char root=ptree[0];
		Tree[cnt]=root;
		if (ptree.length()!=1&&mtree.length()!=1)
		{
			int pos=mtree.find(root);
			work(ptree.substr(1,pos),mtree.substr(0,pos),cnt*2);
			work(ptree.substr(ptree.length()-pos,pos),mtree.substr(mtree.length()-pos,pos),cnt*2+1);
		}
		else
		{
			Tree[cnt]=root;
		}
		
	}
	void trans(int x,node * last,int size)\\将数组变成二叉链表存储
	{
		node *p1;
		if (x==1) 
		{
			hp=(node *)malloc(sizeof(node));
			hp->lson=NULL;
			hp->rson=NULL;
			hp->c=Tree[x];
			if (Tree[x*2]) trans(x*2,hp,1);
			if (Tree[x*2+1]) trans(x*2+1,hp,2);
		}
		else
		{
			p1=(node *) malloc(sizeof(node));
			p1->lson=NULL;
			p1->rson=NULL;
			p1->c=Tree[x];
			if (Tree[x*2]) trans(x*2,p1,1);
			if (Tree[x*2+1]) trans(x*2+1,p1,2);
			if (size==1) last->lson=p1;
			else last->rson=p1;
		}
	}      
	void printlast(node * p)\\后序遍历输出
	{
		
		if (p->lson==NULL && p->rson==NULL) return;
		if (p->lson!=NULL) printpre(p->lson);
		if (p->rson!=NULL) printpre(p->rson);
		cout<<p->c<<" ";
		return;
	}     
	int main()
	{
		cin>>n;
		string pre,mid;
		cin>>pre;
		cin>>mid;
		\\初始化
		work(pre,mid,1);
		trans(1,NULL,0);
		printlast(hp);
	return 0;
	}
\end{lstlisting}
\textbf{样例输入1}\\
ABDECFG\\
DBEAFCG\\
\textbf{样例输出1}\\
DEBFGCA\\
(标准的三层二叉树)
\end{document}
