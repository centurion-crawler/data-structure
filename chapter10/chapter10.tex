\documentclass[UTF8,a4paper]{article}
\usepackage{fancyhdr}
\usepackage{ctex}
\usepackage{CJK}
\usepackage{amsmath}
\usepackage{listings}
\usepackage{graphics}
\usepackage{graphicx}
\usepackage{color}
\usepackage{xcolor}
\usepackage{geometry}
\usepackage{indentfirst}
\setlength{\parindent}{2em}
\geometry{left=1.5cm,right=1.5cm,top=2cm,bottom=1.5cm}
\lstset{breaklines}%这条命令可以让LaTeX自动将长的代码行换行排版
\lstset{extendedchars=false}%这一条命令可以解决代码跨页时,章节标题,页眉等汉字不显示的问题
\lstset{ 
	language=C++,                % choose the language of the code
	basicstyle=\small\sf,    % the size of the fonts that are used for the code
	tabsize=3,                            % sets default tabsize to 3 spaces
	numbers=left,                   % where to put the line-numbers
	numberstyle=\tiny,              % the size of the fonts that are used for the line-numbers
	stepnumber=1,                   % the step between two line-numbers. If it's 1 each line
	% will be numbered
	numbersep=5pt,                  % how far the line-numbers are from the code   %
	keywordstyle=\color[RGB]{33,33,234},               % keywords
	commentstyle=\color[RGB]{0,0,0},    % comments
	stringstyle=\color[rgb]{0.170,0.187,0.102},      % strings
    backgroundcolor=\color{white},
    rulesepcolor=\color[RGB]{20,20,20},  % choose the background color. You must add \usepackage{color}
	showspaces=false,               % show spaces adding particular underscores
	showstringspaces=false,         % underline spaces within strings
	showtabs=false,                 % show tabs within strings adding particular underscores                frame = single,         % adds a frame around the code
	captionpos=b,                   % sets the caption-position to bottom
	breaklines=true,                % sets automatic line breaking
	breakatwhitespace=false,        % sets if automatic breaks should only happen at whitespace
	title=\lstname,                 % show the filename of files included with \lstinputlisting;
	% also try caption instead of title
	mathescape=true,escapechar=?    % escape to latex with ?..?
	escapeinside={\%*}{*)},         % if you want to add a comment within your code
	%columns=fixed,                  % nice spacing
	%morestring=[m]',                % strings
	%morekeywords={%,...},%          % if you want to add more keywords to the set
	%    break,case,catch,continue,elseif,else,end,for,function,global,%
	%    if,otherwise,persistent,return,switch,try,while,...},%
}
\pagestyle{fancy}
\lhead{数据结构作业}
\chead{}
\rhead{\bfseries 22920182204393庄震丰}
\lfoot{}
\cfoot{\thepage}
\rfoot{}
\renewcommand{\headrulewidth}{0.4pt}
\begin{document}
\begin{center}
    \textbf{\LARGE{数据结构作业 第十章——排序}}\\[0.5cm]
    \normalsize{庄震丰 22920182204393}\\[0.3cm]
    \large{Dec. $12^{nd}$, 2019}
\end{center}
\textbf{10-32}\\
荷兰国旗问题:\\
设有一个仅由红、白、蓝三种颜色的条块组成的序列,请编写一个时间复杂度为$O(n)$d的算法,使得这些条块按红、白、蓝的顺序排列,即排成荷兰国旗的图案。\\
\textbf{算法分析}\\
先统计不同颜色的条纹数目,按照数目进行分块排序,先整体排序,把第k2个位置作为轴,将小于蓝色的放到左边,再以k1作为轴,把小于白色的放到左边,则满足红、白、蓝顺序。一共只进行了两趟排序。\\
时间复杂度O(n),空间复杂度O(n).\\
\textbf{code}\\
\begin{lstlisting}
    include<bits/stdc++.h>
    #define maxn 1000
    using namespace std;
    int n;
    int a[maxn];
    void flagsort(int l,int r,int index)
    {
        int i=l,j=r;
        do
        {
            while(a[i]<index) i++;
            while(a[j]>=index) j--;
            if (i<=j)
                {
                    swap(a[i],a[j]);
                    i++;
                    j--;
                }
        }while(i<j);
    }
    void init()
    {
        cin>>n;
        int k1=0,k2=0;
        for (int i=1;i<=n;i++)
        {
            cin>>a[i];
            k1+=(a[i]==0);
            k2+=(a[i]<=1);
        }
        flagsort(1,n,2);
        flagsort(1,k2,1);
    }
    int main()
    {
        init();    
        for (int i=1;i<=n;i++)
            cout<<a[i]<<" ";
    }
\end{lstlisting}
\textbf{10-34}\\
已知$(k_1,k_2,...,k_p)$,是堆,则可以写出一个时间复杂度为$O(logn)$的算法,将$(k_1,k_2,k_3,...,k_{p+1})$调整为堆,试编写一个从p=1起,逐个插入建堆的算法,讨论建堆的复杂度。\\
\textbf{算法分析}\\
每当一个节点进入时,执行insert操作,加到叶子节点,再和其父亲比较,若为大根堆,且当前叶子节点大于其父亲节点,则交换,否则就保持。若为小根堆,当前叶子节点小于其父亲节点,则交换,否则保持。直到插入完所有节点。\\
单个插入时间复杂度O(logn),总时间复杂度O(nlogn),空间复杂度O(n).\\
\text{code}
\begin{lstlisting}
    #include<bits/stdc++.h>
    #define maxn 1000
    using namespace std;
    int n;
    int a[maxn];
    void Insert(int num,int pos)
    {
        a[pos]=num;
        while(a[pos]<a[pos/2]&&pos>=1) 
        {
            swap(a[pos],a[pos/2]);
            pos=pos/2;
        }
    }
    void init()
    {
        cin>>n;
        int k;
        for (int i=1;i<=n;i++)
        {
            cin>>k;
            Insert(k,i);
        }
    }
    int main()
    {
        init();    
        for (int i=1;i<=n;i++)
            cout<<a[i]<<" ";
        return 0;
    }    
\end{lstlisting}
\textbf{10-38}\\
2-路归并排序的另一种策略,先对排序序列扫描一遍,找出划分为若干个最大有序序列,将这些序列作为初始归并段,写出一个算法在链表结构上实现这个策略。\\
\textbf{算法分析}\\
对于给定的列表序列,建立一个循环队列,所有的有序子序列的头节点先进队,之后每次取出两个队尾的头结点指针,将两个子序列分别进行重构,再将其中一个较小的子序列头指针push进队,直到最后只有一个头节点指针。\\
时间复杂度O(logn),空间复杂度O(n).\\
\textbf{Code}
\begin{lstlisting}
    #include <stdio.h>
    void Algo_10_38(LinkList L)
    {
        int k, len;
        int i, m, n;
        int seg[101];							//0号单元计数 
        LinkList SR[101], TR[101];
        LinkList p, r;
        
        for(len=0,p=L->next; p; p=p->next)		//初始化链表指针数组 
            SR[++len] = p;
        
        seg[0] = seg[1] = 1;
        
        for(k=2; k<=len; k++)						//用最大有序子列生成初始归并段 
        {
            if(SR[k]->data<SR[k-1]->data)
            {
                seg[0]++;
                seg[seg[0]] = k;
            }
        }
            
        MSort_10_38(SR, TR, seg, len, 1, seg[0]);	//对各初始归并段的指针排序 
            
        for(k=1,r=L; k<=len; k++)					//根据排好的指针顺序重排记录
        {
            r->next = TR[k];
            r = r->next;
        }
        
        r->next = NULL;
    }
    
    void MSort_10_38(LinkList SR[], LinkList TR[], int seg[], int len, int s, int t)
    {
        int m, ks, ke;
        LinkList R[len+1];
        
        if(s==t)
        {
            ks = seg[s];
            ke = ++s<=seg[0] ? (seg[s]-1) : len;
            for( ; ks<=ke; ks++)
                TR[ks] = SR[ks];
        }
        else
        {
            m = (s+t)/2;
            MSort_10_38(SR, R, seg, len, s, m);
            MSort_10_38(SR, R, seg, len, m+1, t);
            Merge_10_38(R, TR, seg, len, s, m, t);
        }	
    }
    
    void Merge_10_38(LinkList SR[], LinkList TR[], int seg[], int len, int s, int m, int t)
    {
        int k;
        int is, ie, js, je;
        
        is = seg[s];
        ie = seg[m+1]-1;
        js = seg[m+1];
        je = ++t<=seg[0] ? (seg[t]-1) : len;
        
        k = is;
        
        while(is<=ie&&js<=je)
        {
            if(SR[is]->data<=SR[js]->data)
                TR[k++] = SR[is++];
            else
                TR[k++] = SR[js++];
        }
        
        while(is<=ie)
            TR[k++] = SR[is++];
    
        while(js<=je)
            TR[k++] = SR[js++];
    }
    
    void PrintKey(LElemType_L e)
    {
        printf("%d ", e);
    }
    
    int main()
    {	
        FILE *fp;
        LinkList L;
        int num;
            
        printf("创建并输出一个任意序列...\n");
        fp = fopen("Data/Algo_10_37.txt", "r");
        Scanf(fp, "%d", &num);
        CreateList_TL(fp, &L, num);
        ListTraverse_L(L, PrintKey);
        printf("\n\n");	
    
        printf("将关键字按递增顺序排列...\n");
        Algo_10_38(L); 
        ListTraverse_L(L, PrintKey);
        printf("\n\n");	
    
        return 0;
    }
\end{lstlisting}
\textbf{10-42}\\
序列中值序列值的是,如果将此序列排序后,它的第$\frac{n}{2}$个记录,试写出一个中值记录的算法。\\
\textbf{算法分析}\\
利用快速排序的思想,从序列中取一个数mid,然后把序列分成小于等于mid和大于等于mid的两部分,由两个部分的元素个数和k的大小关系可以确定这个数是在哪个部分。对部分序列的探查可以递归处理。
采用分治的思想。时间复杂度$O(n)$,空间复杂度$O(n)$\\
\textbf{code}
\begin{lstlisting}
    #include<bits/stdc++.h>
    #define maxn 1000
    using namespace std;
    int n;
    int a[maxn];
    void flagsort(int l,int r,int rank)
    {
        int i=l,j=r;
        int index=a[(l+r)/2];
        do
        {
            while(a[i]<index) i++;
            while(a[j]>index) j--;
            if (i<=j)
                {
                    swap(a[i],a[j]);
                    i++;
                    j--;
                }
        }while(i<j);
        if(l<=j && rank<=j-l+1) flagsort(l,j,rank);
        if(i<=r && rank>=i-l+1) flagsort(i,r,rank-(i-l));
    }
    void init()
    {
        cin>>n;
        for (int i=1;i<=n;i++)
            cin>>a[i];
        int k=(n+1)/2;
        flagsort(1,n,k);
        cout<<a[k];
    }
    int main()
    {
        init();   
        return 0;
    }
\end{lstlisting}    
\textbf{10-43}\\
已知序列a[1..n]中的关键字各不相同,可按照如下所述实现计数排序,另设一个数组c[1...n],对每一个记录a[i],统计序列中关键字比它小的记录个数存于c[i],则c[i]=0的记录为关键字最小的记录,然后依c[i]值的大小对a中记录进行重新排序,试编写上述算法。\\
\textbf{算法分析}\\
假设输入线性表L的长度为n,$L=L_1,L_2,...,L_n$:线性表的元素属于有限偏序集S,|S|=k且k=O(n),S={S1,S2,..Sk};则计数排序可以描述如下:
1、扫描整个集合S,对每一个Si∈S,找到在线性表L中小于等于Si的元素的个数T(Si);
2、扫描整个线性表L,对L中的每一个元素Li,将Li放在输出线性表的第T(Li)个位置上,并将T(Li)减1。
时间复杂度$O(n)$,空间复杂度$O(n)$。\\
\textbf{code}\\
\begin{lstlisting}
#include <bits/stdc++.h>
using namespace std;
const int MAXN = 100000;
const int k = 1000; 
int a[MAXN], c[MAXN], ranked[MAXN];
 
int main() {
    int n;
    cin >> n;
    for (int i = 0; i < n; ++i) {
        cin >> a[i]; 
        ++c[a[i]];
    }
    for (int i = 1; i < k; ++i)
        c[i] += c[i-1];
    for (int i = n-1; i >= 0; --i)
        ranked[--c[a[i]]] = a[i];
    for (int i = 0; i < n; ++i)
        cout << ranked[i] << endl;
    return 0;
}
\end{lstlisting}
\end{document}
